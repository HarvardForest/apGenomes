% latex table generated in R 3.5.0 by xtable 1.8-2 package
% Thu Apr  5 12:17:36 2018
\begin{table}[ht]
\centering
\begin{tabular}{rrr}
  \hline
 & {\emph{r}} & {\emph{p-value}} \\ 
  \hline
PCQ: Precipitation of Coldest Quarter (BIO19) & 0.344 & 0.050 \\ 
  Longitude & 0.232 & 0.139 \\ 
  MTDQ: Mean Temperature of Driest Quarter (BIO9) & 0.241 & 0.148 \\ 
  Iso: Isothermality (BIO2/BIO7) (* 100) (BIO3) & 0.213 & 0.191 \\ 
  Tmax: Max Temperature of Warmest Month (BIO5) & 0.204 & 0.206 \\ 
  MTWaQ: Mean Temperature of Warmest Quarter (BIO10) & 0.180 & 0.248 \\ 
  PWM: Precipitation of Wettest Month (BIO13) & 0.164 & 0.273 \\ 
  PWeQ: Precipitation of Wettest Quarter (BIO16) & 0.158 & 0.279 \\ 
  MAT: Annual Mean Temperature (BIO1) & 0.165 & 0.292 \\ 
  PA: Annual Precipitation (BIO12) & 0.150 & 0.293 \\ 
  Tmin: Min Temperature of Coldest Month (BIO6) & 0.161 & 0.312 \\ 
  MTCQ: Mean Temperature of Coldest Quarter (BIO11) & 0.150 & 0.327 \\ 
  ATR: Temperature Annual Range (BIO5-BIO6) (BIO7) & 0.128 & 0.387 \\ 
  MTWeQ: Mean Temperature of Wettest Quarter (BIO8) & 0.109 & 0.462 \\ 
  PS: Precipitation Seasonality (Coefficient of Variation) (BIO15) & 0.096 & 0.483 \\ 
  TS: Temperature Seasonality (standard deviation *100) (BIO4) & 0.100 & 0.485 \\ 
  PDM: Precipitation of Driest Month (BIO14) & 0.095 & 0.504 \\ 
  MDR: Mean Diurnal Range (Mean of monthly (max temp - min temp)) (BIO2) & 0.088 & 0.520 \\ 
  PWaQ: Precipitation of Warmest Quarter (BIO18) & 0.083 & 0.532 \\ 
  Latitude & 0.044 & 0.723 \\ 
  PDQ: Precipitation of Driest Quarter (BIO17) & 0.043 & 0.736 \\ 
   \hline
\end{tabular}
\caption{Results of the NMS ordination vector analysis for all whole genome sequences present in NCBI prior to the current sequencing effort.} 
\label{tab:wc_napg_vec}
\end{table}
