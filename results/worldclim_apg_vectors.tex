% latex table generated in R 3.4.2 by xtable 1.8-2 package
% Sun Feb 18 20:11:44 2018
\begin{table}[ht]
\centering
\begin{tabular}{rrr}
  \hline
 & r & p \\ 
  \hline
MDR: Mean Diurnal Range (Mean of monthly (max temp - min temp)) (BIO2) & 0.811 & 0.150 \\ 
  PCQ: Precipitation of Coldest Quarter (BIO19) & 0.742 & 0.333 \\ 
  PA: Annual Precipitation (BIO12) & 0.344 & 0.700 \\ 
  PWM: Precipitation of Wettest Month (BIO13) & 0.332 & 0.700 \\ 
  PWeQ: Precipitation of Wettest Quarter (BIO16) & 0.342 & 0.700 \\ 
  PWaQ: Precipitation of Warmest Quarter (BIO18) & 0.335 & 0.700 \\ 
  TS: Temperature Seasonality (standard deviation *100) (BIO4) & 0.104 & 0.800 \\ 
  Tmin: Min Temperature of Coldest Month (BIO6) & 0.066 & 0.800 \\ 
  ATR: Temperature Annual Range (BIO5-BIO6) (BIO7) & 0.122 & 0.800 \\ 
  PS: Precipitation Seasonality (Coefficient of Variation) (BIO15) & 0.268 & 0.800 \\ 
  Latitude & 0.058 & 0.850 \\ 
  MAT: Annual Mean Temperature (BIO1) & 0.047 & 0.850 \\ 
  Iso: Isothermality (BIO2/BIO7) (* 100) (BIO3) & 0.069 & 0.850 \\ 
  MTCQ: Mean Temperature of Coldest Quarter (BIO11) & 0.058 & 0.850 \\ 
  MTWeQ: Mean Temperature of Wettest Quarter (BIO8) & 0.009 & 0.867 \\ 
  MTWaQ: Mean Temperature of Warmest Quarter (BIO10) & 0.012 & 0.867 \\ 
  PDM: Precipitation of Driest Month (BIO14) & 0.026 & 0.867 \\ 
  Tmax: Max Temperature of Warmest Month (BIO5) & 0.003 & 0.933 \\ 
  PDQ: Precipitation of Driest Quarter (BIO17) & 0.033 & 0.950 \\ 
  Longitude & 0.001 & 1.000 \\ 
  MTDQ: Mean Temperature of Driest Quarter (BIO9) & 0.002 & 1.000 \\ 
   \hline
\end{tabular}
\caption{Results of the NMS ordination vector analysis for only Aphaenogaster spp.} 
\label{tab:wc_apg_vec}
\end{table}
