% latex table generated in R 3.4.4 by xtable 1.8-2 package
% Tue Jul 10 12:09:18 2018
\begin{table}[ht]
\centering
\begin{tabular}{rrr}
  \hline
 & Lon & Lat \\ 
  \hline
{\emph{Acromyrmex echinatior}} & 9.12 & -79.70 \\ 
  {\emph{Atta cephalotes}} & 9.12 & -79.70 \\ 
  {\emph{Atta colombica}} & 9.12 & -79.70 \\ 
  {\emph{Camponotus floridanus}} & 24.62 & -81.54 \\ 
  {\emph{Cardiocondyla obscurior}} & -19.92 & -43.94 \\ 
  {\emph{Cyphomyrmex costatus}} & 9.12 & -79.70 \\ 
  {\emph{Dinoponera quadriceps}} & 9.40 & -79.87 \\ 
  {\emph{Formica selysi}} & 46.58 & 10.32 \\ 
  {\emph{Harpegnathos saltator}} & 15.32 & 75.71 \\ 
  {\emph{Lasius niger}} & 55.76 & 37.62 \\ 
  {\emph{Linepithema humile}} & 37.26 & -122.02 \\ 
  {\emph{Ooceraea biroi}} & 26.21 & 127.68 \\ 
  {\emph{Pogonomyrmex barbatus}} & 20.59 & -100.39 \\ 
  {\emph{Pseudomyrmex gracilis}} & -11.77 & -70.81 \\ 
  {\emph{Solenopsis invicta}} & 33.95 & -83.36 \\ 
  {\emph{Trachymyrmex cornetzi}} & 9.12 & -79.70 \\ 
  {\emph{Trachymyrmex septentrionalis}} & 30.44 & -84.28 \\ 
  {\emph{Trachymyrmex zeteki}} & 9.12 & -79.70 \\ 
  {\emph{Vollenhovia emeryi}} & 20.59 & -100.39 \\ 
   \hline
\end{tabular}
\caption{Coordinates obtained from the literature published for the ant genome sequences used in the biogeographic analyses.} 
\end{table}
