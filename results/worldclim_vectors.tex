% latex table generated in R 3.4.2 by xtable 1.8-2 package
% Thu Feb 15 15:38:59 2018
\begin{table}[ht]
\centering
\begin{tabular}{rrr}
  \hline
 & r & p \\ 
  \hline
PCQ: Precipitation of Coldest Quarter (BIO19) & 0.371 & 0.010 \\ 
  Iso: Isothermality (BIO2/BIO7) (* 100) (BIO3) & 0.364 & 0.016 \\ 
  MTDQ: Mean Temperature of Driest Quarter (BIO9) & 0.247 & 0.065 \\ 
  PWM: Precipitation of Wettest Month (BIO13) & 0.223 & 0.089 \\ 
  PWeQ: Precipitation of Wettest Quarter (BIO16) & 0.216 & 0.098 \\ 
  TS: Temperature Seasonality (standard deviation *100) (BIO4) & 0.205 & 0.118 \\ 
  PS: Precipitation Seasonality (Coefficient of Variation) (BIO15) & 0.194 & 0.134 \\ 
  PA: Annual Precipitation (BIO12) & 0.189 & 0.136 \\ 
  MTCQ: Mean Temperature of Coldest Quarter (BIO11) & 0.189 & 0.137 \\ 
  Latitude & 0.182 & 0.148 \\ 
  Tmin: Min Temperature of Coldest Month (BIO6) & 0.175 & 0.159 \\ 
  PDM: Precipitation of Driest Month (BIO14) & 0.172 & 0.170 \\ 
  MAT: Annual Mean Temperature (BIO1) & 0.161 & 0.190 \\ 
  ATR: Temperature Annual Range (BIO5-BIO6) (BIO7) & 0.143 & 0.225 \\ 
  Tmax: Max Temperature of Warmest Month (BIO5) & 0.136 & 0.258 \\ 
  MDR: Mean Diurnal Range (Mean of monthly (max temp - min temp)) (BIO2) & 0.123 & 0.283 \\ 
  Longitude & 0.116 & 0.317 \\ 
  PDQ: Precipitation of Driest Quarter (BIO17) & 0.106 & 0.352 \\ 
  MTWaQ: Mean Temperature of Warmest Quarter (BIO10) & 0.096 & 0.388 \\ 
  MTWeQ: Mean Temperature of Wettest Quarter (BIO8) & 0.074 & 0.478 \\ 
  PWaQ: Precipitation of Warmest Quarter (BIO18) & 0.055 & 0.595 \\ 
   \hline
\end{tabular}
\caption{Results of the NMS ordination vector analysis.} 
\label{tab:wc_vec}
\end{table}
