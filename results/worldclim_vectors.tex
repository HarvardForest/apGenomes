% latex table generated in R 3.4.2 by xtable 1.8-2 package
% Tue Mar 20 17:26:06 2018
\begin{table}[ht]
\centering
\begin{tabular}{rrr}
  \hline
 & r & p \\ 
  \hline
PCQ: Precipitation of Coldest Quarter (BIO19) & 0.406 & 0.004 \\ 
  Iso: Isothermality (BIO2/BIO7) (* 100) (BIO3) & 0.386 & 0.005 \\ 
  MTDQ: Mean Temperature of Driest Quarter (BIO9) & 0.307 & 0.019 \\ 
  PWM: Precipitation of Wettest Month (BIO13) & 0.285 & 0.029 \\ 
  PWeQ: Precipitation of Wettest Quarter (BIO16) & 0.279 & 0.032 \\ 
  Tmin: Min Temperature of Coldest Month (BIO6) & 0.276 & 0.033 \\ 
  MAT: Annual Mean Temperature (BIO1) & 0.265 & 0.037 \\ 
  MTCQ: Mean Temperature of Coldest Quarter (BIO11) & 0.263 & 0.039 \\ 
  ATR: Temperature Annual Range (BIO5-BIO6) (BIO7) & 0.237 & 0.058 \\ 
  PA: Annual Precipitation (BIO12) & 0.241 & 0.058 \\ 
  MTWaQ: Mean Temperature of Warmest Quarter (BIO10) & 0.236 & 0.059 \\ 
  TS: Temperature Seasonality (standard deviation *100) (BIO4) & 0.232 & 0.061 \\ 
  Longitude & 0.211 & 0.081 \\ 
  PDM: Precipitation of Driest Month (BIO14) & 0.198 & 0.098 \\ 
  Latitude & 0.188 & 0.111 \\ 
  MTWeQ: Mean Temperature of Wettest Quarter (BIO8) & 0.153 & 0.173 \\ 
  PDQ: Precipitation of Driest Quarter (BIO17) & 0.132 & 0.223 \\ 
  Tmax: Max Temperature of Warmest Month (BIO5) & 0.133 & 0.225 \\ 
  PS: Precipitation Seasonality (Coefficient of Variation) (BIO15) & 0.106 & 0.305 \\ 
  MDR: Mean Diurnal Range (Mean of monthly (max temp - min temp)) (BIO2) & 0.082 & 0.409 \\ 
  PWaQ: Precipitation of Warmest Quarter (BIO18) & 0.038 & 0.671 \\ 
   \hline
\end{tabular}
\caption{Results of the NMS ordination vector analysis.} 
\label{tab:wc_vec}
\end{table}
